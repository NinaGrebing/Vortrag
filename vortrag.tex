\documentclass[presentation,table]{beamer}
%Fuer Erzeugung der Handouts:
%\documentclass[handout]{beamer}

\usepackage{ifthen}
\newboolean{presentation} %Deklaration einer boolschen Variable
\setboolean{presentation}{true}

\mode<handout>{
	\usepackage{pgf}
	\usepackage{pgfpages}
	\pgfpagesuselayout{4 on 1}[a4paper,border shrink=5mm, landscape]	
}
\mode<presentation>{
	\usetheme{Juelich}
}

%Fuer farbige Texte
\usepackage{color}
\definecolor{darkgreen}{rgb}{0,0.5,0}
\definecolor{middlegray}{rgb}{0.5,0.5,0.5}
 \definecolor{lightgray}{rgb}{0.8,0.8,0.8}
 \definecolor{orange}{rgb}{0.8,0.3,0.3}
 \definecolor{yac}{rgb}{0.6,0.6,0.1}
 \definecolor{lred}{rgb}{1,0.7,0.7}
 \definecolor{lgreen}{rgb}{0.7,1,0.7}
\newcommand{\minus}{{\mbox{\begin{large}\color{red}\textbf{--}\color{black}\end{large}} }}
\newcommand{\plus}{{\mbox{\begin{large}\color{darkgreen}\textbf{+}\color{black}\end{large}} }}

\setbeamertemplate{slide counter}[showall][]

\newcommand{\noheader}{
\setbeamertemplate{headline}
{%
\begin{beamercolorbox}[leftskip=1em,ht=2.5mm,dp=0mm]{header}
\end{beamercolorbox}
\begin{beamercolorbox}[leftskip=1em,ht=7.5mm,dp=0mm]{header}
\end{beamercolorbox}
}
}

\newcommand{\makeheader}{
\setbeamertemplate{headline}
{
  \begin{beamercolorbox}[leftskip=1em,ht=2.5mm,dp=0mm]{header}
    \usebeamercolor{fzjblue50}\colorbox{bg}{ \color{white}\textbf{\thepart.~\insertpart}}
  \end{beamercolorbox}
  \begin{beamercolorbox}[leftskip=1em,ht=7.5mm,dp=0mm]{header}
  \end{beamercolorbox}
}
}

\newcommand{\mymakepart}{
\noheader
\makepart
\makeheader
}

\usepackage{listings}

\usepackage{helvet}
%Deutsche Pakete, deutsche Worte fuer Inhaltsverzeichnis und Zusammenfassung
\usepackage[ngerman]{babel}
\usepackage[latin1]{inputenc}
\usepackage[T1]{fontenc}
\usepackage{ucs}
\usepackage{lmodern}
%Fuer Einbinden von Grafiken
\usepackage{graphicx}
\usepackage{listings}
\usepackage{longtable}


%\title{\Large Vergleich zwischen Vizard VR Toolkit und \mbox{Unreal Engine 4} als Plattformen fuer virtuelle \\Experimente im Bereich der Fussgaengerdynamik unter Nutzung der Oculus Rift}
\subtitle{\Large Entwicklung eines Softwaresystems zur Carsharing Simulation}
\author{Nina Grebing}
\institute{Institut f�r Bio- und Geowissenschaften}
\date{23. August 2017}

\begin{document}

\noheader 

\maketitle 

\begin{frame}
	\frametitle{Inhalt}
	\begin{enumerate}
		\item \hyperlink{vorstellung}{Vorstellung}
		\item \hyperlink{aufgabenanalyse}{Aufgabenanalyse}
		\item \hyperlink{algorithmus}{Algorithmus}		
		\item \hyperlink{konzept}{Programmkonzept}
		\item \hyperlink{tests}{Tests}
		\item \hyperlink{fazit}{Fazit}
	\end{enumerate}
\end{frame}

\part{Vorstellung}
\label{vorstellung}
\mymakepart

\begin{frame}
	\frametitle{Forschungszentrum J�lich - IBG-1}
	\begin{minipage}[l]{0.5\textwidth}
		\begin{itemize}
			\item Forschung in den Bereichen Energie, Umwelt, Medizin, Simulationswissenschaften und Nanotechnologie
			\item 10 Institute (�ber 60 Institutsbereiche)
			\item �ber 5000 Mitarbeiter
			\item Institut f�r Bio- und Geowissenschaften
		\end{itemize}
	\end{minipage}
	\begin{minipage}[l]{0.5\textwidth}
		\begin{figure}
%			\includegraphics[height=4cm]{./../daten/fzj.png}
		\end{figure}
	\end{minipage}	
\end{frame}

\begin{frame}
	\frametitle{Praktische Arbeit}
	\begin{block}{Aufgaben w�hrend der Ausbildungszeit}
		\begin{itemize}
			\item Entwicklung eines Konzeptes zur Normalisierung einer Datenbank
			\item Entwurf und Implementierung von PostgreSQL-Datenbanken
			\item Konvertierung von MSSQLServer-Datenbanken nach PostgreSQL
			\item Entwicklung von Webanwendungen
		\end{itemize}
	\end{block}
\end{frame}

\part{Aufgabenanalyse}
\label{aufgabenanalyse}
\mymakepart

\begin{frame}
	\frametitle{Aufgabenstellung}
	\begin{itemize}
		\item Simulation und Auswertung des Kundenverhaltens eines Carsharing Unternehmens
		\item Stadt in m $\times$ m Gebiete unterteilt
		\item Verlauf von Nachfrage und Abstellung der Autos f�r jedes Gebiet als monoton wachsendes Polynom maximal 4. Grades, positiv auf dem Intervall [0;24]
		\item Einlesen der Daten aus einer Datei
		\item 
	\end{itemize}
	 
	
\end{frame}


\begin{frame}
	\frametitle{Aufgabenstellung}
	\begin{itemize}
		\item Bestimmen des maximalen Bedarfs und des Endzustandes
	\end{itemize}
	\begin{block}{Pr�zisierungen}
		\begin{itemize}
			\item Minimum 1 Quadrant
			\item Maximum 10 Quadranten
			\item Abbruchgenauigkeit der Bisektion: $e < 10^{-4}$
		\end{itemize}
	\end{block}
	
\end{frame}

\part{Algorithmus}
\label{algorithmus}
\mymakepart

\begin{frame}[fragile]
	\frametitle{Programmablauf}
	\begin{columns}
		%linke Spalte
		\column{5cm}
		\begin{itemize}[<+->]
			\item Einlesen der Daten
			\item Erzeugen der Teilgebiete mit den jeweiligen Polynomen
			\item Erzeugen der Simulation mit den Teilgebieten
			\item Berechnung des Bedarf f�r jedes Teilgebiet
			\item Ausgabe der Ergebnisse
		\end{itemize}
		
		
		
		\lstset{
		frame=single,
		basicstyle=\tiny\ttfamily,
		linewidth=\textwidth,
		numbers=none
		}
		
		%rechte Spalte
		\column{5cm}
		\begin{onlyenv}<1>
			\begin{lstlisting}
			fdsfsf
			\end{lstlisting}
		\end{onlyenv}
		\only<2>{
			\begin{itemize}
				\item 
			\end{itemize}
		}
		\only<3>{
			\begin{itemize}
				\item 
			\end{itemize}
		}
		\only<4>{
			\begin{itemize}
				\item 
			\end{itemize}
		}
		\begin{onlyenv}<5>
			\begin{lstlisting}
			
			\end{lstlisting}
		\end{onlyenv}
		
	\end{columns}
\end{frame}



\part{Programmkonzept}
\label{konzept}
\mymakepart

\begin{frame}
	\frametitle{UML-Diagramm}
	
	
\end{frame}


\begin{frame}
	\frametitle{}
	
	
\end{frame}




\part{Test}
\label{test}
\mymakepart

\begin{frame}
	\frametitle{Teststrategie}
	
	
\end{frame}

\begin{frame}
	\frametitle{Testklassen}
	
	
\end{frame}

\part{Fazit}
\label{fazit}
\mymakepart

\begin{frame}
	\frametitle{Ausblick}
	
	
\end{frame}


\begin{frame}[plain]
	\LARGE\textcolor{fzjblue50}{Vielen Dank f�r Ihre Aufmerksamkeit!}
\end{frame}


\setbeamertemplate{footer element3}{\empty}
\setbeamertemplate{footer element4}{}

\part{Zusatzinformationen}
\label{zusatz}
\mymakepart

\begin{frame}
	\frametitle{Sequenzdiagramm}

\end{frame}


\addtocounter{framenumber}{-3}

\end{document}